\section*{Постановка задачи}
\addcontentsline{toc}{section}{Постановка задачи}

\textit{Цель задачи:} создать программу NetFlow-сенсор, поддерживающую NetFlow export protocol версии 9.

\linespace

\textit{Требования:}
\begin{enumerate}[wide, noitemsep]
    \item ПО должно работать на ПК под управлением Debian GNU\textbackslash Linux (версии 10 и новее).
    \item Для реализации использовать язык программирования C.
    \item Сборка должна осуществляться GNU Toolchain.
    \item Дистрибуция должна осуществляться при помощи deb-пакета.
    \item Сенсор должен иметь следующие параметры запуска:
    \begin{enumerate}[wide=\dimexpr\parindent+1.25cm, noitemsep]
        \item имя сетевого интерфейса, на котором вести учёт трафика
        \item IP-адрес и номер UDP-порта хоста с NetFlow-коллектором, куда отправлять пакеты с данными по статистике, формат IP:порт (например, 192.168.0.2:9995)
    \end{enumerate}
    \item Основной приоритет при разработке: постараться максимизировать нагрузку, которую может обработать программа, не пропуская пакеты.
\end{enumerate}

\linespace

Описание протокола NetFlowV9: \url{https://www.ietf.org/rfc/rfc3954.txt}

\linespace

Определение потока: поток (flow) это пакеты, имеющие одинаковые поля:
\begin{itemize}[wide, noitemsep]
    \item IP адрес источника;
    \item IP адрес получателя;
    \item Для TCP/UDP пакетов:
        \begin{itemize}[wide=\dimexpr\parindent+1.25cm, noitemsep]
            \item TCP/UDP порт источника;
            \item TCP/UDP порт получателя;
        \end{itemize}
    \item Для ICMP пакетов:
        \begin{itemize}[wide=\dimexpr\parindent+1.25cm, noitemsep]
            \item ICMP код;
            \item ICMP тип;
        \end{itemize}
    \item Протокол L4 (поле IP Protocol Number);
    \item IP ToS.
\end{itemize}        

\linespace

Сенсор должен запускаться с указанием сетевого интерфейса, на котором вести учёт трафика, без подтверждения пользователя приступать к учёту и вести его до прерывания работы пользователем.

\linespace

Пакет NetFlow должен содержать следующий набор полей:
\begin{itemize}[wide, noitemsep]
    \item IN\_BYTES,
    \item IN\_PKTS,
    \item FLOWS,
    \item PROTOCOL,
    \item SRC\_TOS,
    \item TCP\_FLAGS,
    \item L4\_SRC\_PORT,
    \item IPV4\_SRC\_ADDR,
    \item INPUT\_SNMP,
    \item L4\_DST\_PORT,
    \item IPV4\_DST\_ADDR,
    \item LAST\_SWITCHED,
    \item FIRST\_SWITCHED,
    \item ICMP\_TYPE,
    \item FLOW\_ACTIVE\_TIMEOUT,
    \item FLOW\_INACTIVE\_TIMEOUT,
    \item IPV4\_IDENT,
    \item IN\_SRC\_MAC,
    \item IN\_DST\_MAC,
    \item IF\_NAME.
\end{itemize}

\newpage

\textit{Проверка работы}

\linespace

Сенсор отправляет данные в коллектор. В качестве коллектора для тестирования можно использовать, например, \verb|nfcapd| из пакета \verb|nfdump|. Отправять отладочный трафик на интерфейс можно при помощи утилиты \verb|tcpreplay| из одноимённого пакета. Для генерации дампа отладочного трафика можно использовать: \url{https://github.com/cslev/pcap_generator}.

\linespace

Вместе с кодом нужно предоставить информацию о максимальной нагрузке, которую может выдержать предложенная реализация.

\linespace

\textit{Рекомендации}

\linespace

Пример проекта, который можно из исходников собрать в deb-пакет:\\
\url{https://gitlab.com/vgeo89/examples/-/tree/main/deb_package}

\linespace

\textit{Информация}

\linespace

\begin{itemize}[wide, noitemsep]
    \item \url{http://xgu.ru/wiki/NetFlow}
    \item \url{https://www.ietf.org/rfc/rfc3954.txt}
    \item \url{https://www.cisco.com/en/US/technologies/tk648/tk362/technologies_white_paper09186a00800a3db9.html}
    \item \url{https://habr.com/ru/company/metrotek/blog/327894/}
\end{itemize}
