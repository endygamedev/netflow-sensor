\section*{Приложение А}
\addcontentsline{toc}{section}{Приложение А}

\begin{center}
    \textbf{Coding Style}
\end{center}

\section*{C}

В сторонних проектах с собственным описанным стилем оформления кода следует придерживаться правил этого стиля:
\begin{itemize}
    \item U-Boot: \url{https://www.denx.de/wiki/U-Boot/CodingStyle}
    \item Linux: \url{https://www.kernel.org/doc/html/v4.10/process/coding-style.html}
\end{itemize}

\linespace

В сторонних проектах без описанного стиля следует оформлять наш код по аналогии с остальными исходниками проекта.

\linespace

В наших собственных проектах за основу взят стиль, принятый в linux с некоторыми изменениями.

\subsection*{Отступы}

В качестве отступа используется один таб шириной в четыре пробела.
\begin{lstlisting}[language=C]
int main(void)
{
    int a = 0;
    return a;
}
\end{lstlisting}

Лишние табы и пробелы в конце строк следует удалять.

\subsection*{Длина строк}

Следует избегать превышения ограничения в 80 символов на строку. Если выражение не помещается в 80 символов, его следует разделить на части. При этом желательно, чтобы каждый аргумент функции находился в отдельной строке. Пример:
\begin{lstlisting}[language=C]
return_code = my_structure->callback_with_arguments(argument_number_1,
                                                         argument_number_2);
\end{lstlisting}

Строки, выводимые программой на экран, не следует сокращать при превышении ими ограничения.

\subsection*{Скобки}

В условиях, циклах и объявлениях структур открывающая фигурная скобка не переносится на следующую строку. Закрывающая переносится:
\begin{lstlisting}[language=C]
while (a > b) {
    a--;
    b++;
    if (a == b) {
        do_something();
    }
}
\end{lstlisting}

Закрывающая фигурная скобка располагается на том же уровне отступов, что и начало всей конструкции.

\linespace

В объявлениях функций открывающая фигурная скобка переносится на следующую строку:
\begin{lstlisting}[language=C]
int main(int argc, char **argv)
{
    return 0;
}
\end{lstlisting}

В условиях и циклах перед открывающей круглой скобкой ставится пробел. После закрывающей скобки ставится пробел:

\newpage

\begin{lstlisting}[language=C]
if (a == 0) {
    return a;
}
\end{lstlisting}

Фигурные скобки должны присутствовать даже если в блоке всего одно выражение.

\subsection*{Пробелы}

Бинарные и тернарные операторы окружаются пробелами:
\begin{lstlisting}[language=C]
= + - < > * / % | & ^ <= >= == != ? :
\end{lstlisting}

После унарных операторов не ставится пробел:
\begin{lstlisting}[language=C]
& * + - ~ ! sizeof typeof alignof __attribute__ defined
\end{lstlisting}

Пробел не ставится после префиксных инкремента и декремента:
\begin{lstlisting}[language=C]
++i
--i
\end{lstlisting}
и перед постфиксными:
\begin{lstlisting}[language=C]
i++
i--
\end{lstlisting}

Пробелы не ставятся вокруг операторов структур:
\begin{lstlisting}[language=C]
my_struct.element
my_struct->element
\end{lstlisting}

При объявлении указателя оператор \verb|*| идет перед именем указателя, а не после его типа:
\begin{lstlisting}[language=C]
void *buffer
\end{lstlisting}

\subsection*{Имена и объявления}
В именовании функций, переменных и типов используется \textit{snake\_case}. Примеры:
\begin{lstlisting}[language=C]
void my_function(int my_argument);
struct my_struct *s;
\end{lstlisting}

В именах желательно стараться избегать сокращений, из-за которых становится не ясно назначение сущности. Например, вместо \textbf{dctl()} использовать более полное имя \textbf{device\_control()}.

\linespace

Объявлять переменные в функции желательно рядом с выражениями, их использующими, если функция достаточно длинная (от 10 строк), либо в начале функции, если короткая (до 10 строк). Например, переменную-счётчик для цикла for следует объявлять так:
\begin{lstlisting}[language=C]
for (int i = 0; i < 3; ++i) {
    actions();
}
\end{lstlisting}

Также нужно следить за количеством переменных в одной функции. Если их число превышает пять, то следует задуматься о декомпозиции функции.

\subsection*{Функции}

Функции должны быть как можно короче и выполнять как можно меньше действий. Длинные функции следует разбивать на подфункции, действия внутри которых связаны по смыслу. Пример:
\begin{lstlisting}[language=C]
int collect_data_and_calculate_result(struct program_context *c)
{
    void *data = collect_data(c);

    return calculate_result(data);
}

int init_and_run_program(struct arguments *a){
    struct program_context *c = init_context(a);

    return collect_data_and_calculate_result(c);
}
\end{lstlisting}

\subsection*{Структуры перечисления}

Их объявления должны выглядеть следующим образом.

Структуры только с именем:
\begin{lstlisting}[language=C]
struct my_struct_name {
    int i;
    int j;
    void *buffer;
};
\end{lstlisting}

Структуры с typedef должны иметь суффикс \_t в имени типа:
\begin{lstlisting}[language=C]
typedef struct my_struct {
    int i;
    int j;
    void *buffer;
} my_struct_t;
\end{lstlisting}

Перечисление объявляются аналогичным образом. Все варианты перечисления должны именоваться в верхнем регистре:
\begin{lstlisting}[language=C]
typedef enum {
    ADDRESS_7BIT,
    ADDRESS_10BIT,
} addressing_mode_t;
\end{lstlisting}

\subsection*{Макросы}
Желательно избегать написания макросов. Особенно вложенных, так как это приводит к сложностям в отладке.

\linespace

Макросы именуются с использованием только верхнего регистра. Макрос и отдельно его входные параметры должны быть окружены круглыми скобками:
\begin{lstlisting}[language=C]
#define MIN(x, y)       ((x) < (y) ? (x) : (y))
\end{lstlisting}

Макросы с несколькими выражениями должны быть заключены в блок \verb|do {} while(0)|:
\begin{lstlisting}[language=C]
#define DO_ACTION(a, b)                                 \
                    do {                                \
                        if (do_first(a) >= 0) {         \
                        do_next(b);                     \
                        }                               \
                    } while(0)                          \
\end{lstlisting}

\subsection*{Использование goto}

\verb|goto| можно использовать для обработки ошибок. Переходя должны осуществляться только в пределах одной функции. Пример:
\begin{lstlisting}[language=C]
int configure_device(void)
{
    int ret = 0;
    struct my_device *d = malloc(sizeof(my_device));
    if (!d) {
        return -ENOMEM;
    }

    ret = init_stage_first(d);
    if (ret != 0) {
        goto err_free;
    }
    
    ret = init_stage_last(d);
    if (ret != 0) {
        goto err_free;
    }

    return 0;

err_deinit:
    deinit_stage_first(d);
err_free:
    free(d);
    return ret;
}
\end{lstlisting}

\subsection*{Комментарии}

Комментарии следует использовать только в качестве документации и для пояснения каких-то не очевидных специфичных случаев.

\subsubsection*{Функции}

Документировать следует функции, поведение, входные и выходные параметры которые не очевидны. Пример:
\begin{lstlisting}[language=C]
/*
 * Configures an I2C bus.
 *
 * addr_lenght should be ADDRESS_7BIT or ADDRESS_10BIT.
 * speed should be SPEED_100MHZ or SPEED_400MHZ.*
 * Returns 0 on success, -1 otherwise.
 */
int configure_i2c_bus(int bus_number, int addr_length, int speed)
\end{lstlisting}

Не следует документировать очевидные функции. Например, в достаточно очевидны назначение и принцип работы функции. Пример:
\begin{lstlisting}[language=C]
int sum_int(int a, int b)
{
    return a + b;
}
\end{lstlisting}

\subsubsection*{Магические числа}

Если назначение числового значения не очевидно из имени переменной или макроса, которым оно присваивается, то это значение следует снабдить комментарием. Пример:

\begin{lstlisting}[language=C]
#define DEVICE_CONTROL_REGISTER_ADDR 0x4
/* Set required device mode on startup. See device datasheet page 100, table 20. */
#define DONTROL_REGISTER_DATA 0x4f7a
\end{lstlisting}

\subsection*{Заголовочные файлы}

Должны обязательно содержать защиту от повторного включения:
\begin{lstlisting}[language=C]
#ifndef MY_HEADER_H
#define MY_HEADER_H
All the contents of the header file here
#endif
\end{lstlisting}

либо

\begin{lstlisting}[language=C]
#pragma once
Content
\end{lstlisting}

Содержимое может включать:
\begin{itemize}
    \item дополнительные \verb|include|
    \item объявление макросов
    \item объявление констант
    \item прототипы функций
\end{itemize}

но реализации функций должны содержаться только в файлах \verb|.c|

\newpage

\section*{Python}
См. \url{https://peps.python.org/pep-0008/}
